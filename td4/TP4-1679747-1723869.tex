\documentclass[11pt,letterpaper]{article}

%Langue et caractères spéciaux
\usepackage[french]{babel} 
\usepackage[T1]{fontenc}
\usepackage{lmodern}
\usepackage[utf8]{inputenc}
\usepackage{textcomp}
\usepackage{caption}
\captionsetup[table]{name=\textsc{Tableau}}
%Marges
\usepackage[top=2cm, bottom=2cm, left=2cm, right=2cm, columnsep=20pt]{geometry}
\usepackage{scrextend}
\setlength\parindent{0pt}
%Math
\usepackage{amsmath}
%Graphiques
\usepackage{graphicx}
\usepackage{float}
\usepackage{hyperref}
\hypersetup{
	colorlinks=true,
	urlcolor=black,
	linkcolor=black,
	citecolor=black}
\usepackage{listings}
\usepackage{color}
\definecolor{dkgreen}{rgb}{0,0.6,0}
\definecolor{mauve}{rgb}{0.58,0,0.82}
\lstdefinestyle{sql}{
  language=SQL,
  frame=single,
  aboveskip=3mm,
  belowskip=3mm,
  showstringspaces=false,
  columns=flexible,
  basicstyle={\footnotesize\ttfamily},
  numberstyle={\tiny},
  numbers=left,
  keywordstyle=\color{blue},
  commentstyle=\color{dkgreen},
  stringstyle=\color{mauve},
  breaklines=true,
  breakatwhitespace=true,
  tabsize=4,
  otherkeywords={=, REPLACE, AFTER, BEGIN, if, DECLARE}
}
\renewcommand{\lstlistingname}{Déclencheur}
\renewcommand{\thesubsection}{\thesection.\Alph{subsection})}

%%%%%%%%%%%%
% Document %
%%%%%%%%%%%%
\begin{document}

%%%%%%%%%%%%%%
% Page titre %
%%%%%%%%%%%%%%
\begin{titlepage}
\begin{center}

\includegraphics[width=0.25\textwidth]{./logo.png}~\\[1cm]

\textsc{\huge École Polytechnique de Montréal}\\[1.5cm]

\rule{0.5\linewidth}{0.5mm} \\[0.4cm]
{\LARGE TP4 - Automne 2015}\\[0.4cm]
{\Large JDBC++, déclencheurs (triggers)}\\[1.0cm]

{\large INF3710}\\[0.4cm]
{\large \textbf{Bases de données}}\\[0.4cm]

\rule{0.5\linewidth}{0.5mm} \\[1.0cm]

{\large par}\\[0.6cm]
\begin{Large}
  \begin{tabular}{r l}
    Louis \textsc{Racicot} & 1679747\\[0.4cm]
    Samuel \textsc{Rondeau} & 1723869\\[0.4cm]
  \end{tabular}
\end{Large}


\vfill

{\large Remis le}\\[0.3cm]
{\Large \today}\\[1.5cm]
{\large Richard Jaramillo, Département de génie informatique et génie logiciel}\\[0.3cm]
{\large École Polytechnique de Montréal}

\end{center}
\end{titlepage}



\newpage
\section{Modifications de la base de données}
\hrule
\vspace{1em}
Afin de répondre aux nouvelles demandes et de produire une base de données représentative de la situation, 
voici les modifications effectuées par rapport à la base de données du travail pratique précédent.\\
\renewcommand{\labelitemi}{$\bullet$}
\begin{itemize}
  \item On crée le champ \textbf{nbInvite}, le nombre d’invités, à la table \textbf{Inscription};
  \item On ajoute une vérification (\textit{CHECK}) sur ce nombre;
  \item On crée la table \textbf{Attente}, où chaque entrée est une inscription non confirmée en attente;
  \item On ajoute à la table \textbf{Sortie} les champs \textbf{description}, \textbf{adresse} et \textbf{nbMax};
  \item On crée la table \textbf{Commentaire} avec les champs \textbf{idComment}, \textbf{pseudo}, \textbf{idSort}, \textbf{text} et \textbf{dte}.
\end{itemize}


\hfill
\section{Contraintes d´intégrité}
\hrule
\vspace{1em}

\subsection{Inscription de l’organisateur d’une sortie}
\begin{lstlisting}[style=sql, label={lst:p3}]
CREATE OR REPLACE TRIGGER inscrire_organisateur
AFTER INSERT ON Sortie
FOR EACH ROW

BEGIN
	INSERT INTO Inscription
	(pseudo, idSort, statut)
	VALUES
	(:NEW.responsable, :NEW.idSort, 0);
END;
\end{lstlisting}


\subsection{Inscription à deux sorties simultanées}
\begin{lstlisting}[style=sql, label={lst:p4}]
CREATE OR REPLACE TRIGGER sorties_simultanees
BEFORE INSERT ON Inscription
FOR EACH ROW

DECLARE
  newDate DATE;
  nb INTEGER;

BEGIN
  SELECT dte INTO newDate
  FROM Sortie
  WHERE idSort = :NEW.idSort;
  
  SELECT COUNT(*) INTO nb
  FROM SORTIE S
  WHERE
    extract(HOUR FROM (S.dte - newDate)) < 4 AND
    extract(HOUR FROM (S.dte - newDate)) > -4 AND
    extract(DAY FROM (S.dte - newDate)) = 0
  ;

	if nb > 1 then
	RAISE_APPLICATION_ERROR(-20000, 'Sorties concurrentes');
	end if;
END;
\end{lstlisting}


\subsection{Propriétés 3 et 4}
\subsubsection*{Nombre d’invités}
Avant chaque insertion dans la table Inscription, si le nombre d’invités (excluant le membre) à l’inscription est supérieur à 7, on rejette l’insertion.

\subsubsection*{Inscription après la date}
Avant chaque insertion dans la table Inscription, on récupère l’ID de la sortie, on sélectionne la sortie avec l’ID correspondant dans la table Sortie, et on compare la date de l’événement avec la date actuelle. Si la date de la sortie est antérieure à la date actuelle, on rejette l’insertion.


\subsection{Autres propriétés}
D'autres propriétés pourraient nécessiter des déclencheurs pour assurer leur intégrité, telles que\\
\begin{itemize}
  \item Limiter le nombre d'invités par membre, par exemple, limiter à 2 invités (en plus du membre) pour une sortie à places limitées. Ainsi, les places disponibles pour une sortie ne pourront être monopolisées par un membre et ses amis.
  \item Empêcher de modifier une sortie ayant déjà eu lieu ou une inscription à une telle sortie. Dans le même ordre d'idée, empêcher le devancement d'une sortie à une date antérieure à la date d'aujourd'hui.
  \item Empêcher le changement d'heure d'une sortie dans les 24 heures avant qu'elle ait normalement lieu, ou son devancement de sorte qu'elle ait lieu subitement dans les 24 prochaines heures. Par exemple, si nous somme le 1$^{er}$ mars à 18h, la sortie $A$ prévue pour le 2 mars à 12h ne pourrait être devancée ou retardée. La sortie $B$ prévue le 3 mars à 12h ne pourrait être devancé avant le 2 mars à 12h.
\end{itemize}



\section{Implantation des méthodes en utilisant JDBC}
\hrule
\vspace{1em}
\stepcounter{subsection}
\subsection{Captures d'écran}
\begin{figure}[H]
  \centering
  \includegraphics[scale=0.5]{snapshot1.png}
  \caption{Page d'accueil et menu principal.}
\end{figure}

\begin{figure}[H]
  \centering
  \includegraphics[scale=0.5]{snapshot2.png}
  \caption{Création d'une nouvelle sortie et mise à jour.}
\end{figure}

\begin{figure}[H]
  \centering
  \includegraphics[scale=0.5]{snapshot3.png}
  \caption{Liste des sorties à venir.}
\end{figure}

\begin{figure}[H]
  \centering
  \includegraphics[scale=0.5]{snapshot5.png}
  \caption{Vue détaillée d'une sortie.}
\end{figure}

\begin{figure}[H]
  \centering
  \includegraphics[scale=0.5]{snapshot4.png}
  \caption{Changement d'utilisateur.}
\end{figure}

\end{document}
